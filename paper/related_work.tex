\section{Background and Related Work}
\label{sec:background}
The work we will present in this paper relates to the topics of privacy-preserving process mining, cross-organizational process mining  and secure multi-party computation. In the remainder of this section we will introduce related work out of these areas of research.
\subsection{Privacy-preserving Process Mining}
\label{sec:privacy_process_mining}
In the area of privacy-preserving process mining\cite{pika2019towards} two general approaches have been established\cite{caise/Fahrenkrog-Petersen19}: anonymizing the event data and directly performing privatized process mining techniques. For the anonymization of event logs from one organization techniques such as \emph{PRETSA}\cite{icpm/Fahrenkrog-Petersen19}, an algorithm to ensure the established privacy metrics $k$-anonymity\cite{sweeney2002k} and $t$-closeness\cite{li2007t}, exist. These metrics\cite{DBLP:journals/csur/WagnerE18}, based on data similarity, are widely adopted and offer protection against certain attacks, like the disclosure of the identity of individuals involved in the dataset.  The necessity of privacy-preserving process mining, due to novel legal development such as the GDPR, was recently discussed in \cite{mannhardt2018privacy}.
 Approaches based on cryptography\cite{burattin2015toward,simpda/RafieiWA18} have also been proposed. \\
 On the side of privatized process mining methods for prrivacy-preserving process discovery\cite{MannhardtKBWM19,tillem2016privacy,tillem2017mining} and the privacy-aware discovery of resource roles\cite{rafiei2019role} have been established. Recently the 
 Both privatized process mining and techniques for anonymizing event logs have been made available for a large audience with the tool ELPaaS\cite{elpaas2019}. The tool and all techniques mentioned above, with the exception of \cite{tillem2017mining}, a concerned with privacy-preserving process mining for one organization and have not yet adopted or evaluated for a cross-organizational setting. While is focussed on a cross-organizational setting \cite{tillem2017mining} it is only focused on  generating process models, while our aim is answering a wide range of queries about a business process.
 

 \subsection{Cross-Organizational Process Mining}
 \label{sec:cross_orgranizational}
The specific problem of privacy in cross-organizational process mining has been addressed in \cite{Liu19}. Liu et. al provide a framework for privacy-preserving cross-organizational process mining based on the assumption that a trusted third party exists and mostly focus on the discovery of process models. Our plan is to provide a solution for scenarios without a trusted third party and lays the focus on computing queries over distributed event logs in a privacy-preserving manner. Other approaches of cross-organizational process mining, that do not consider privacy aspects are presented in \cite{schulz2004facilitating,zeng2013cross}. Such as \cite{Liu19} it also aiming in the discovery of process models.
Another line for research\cite{aksu2016cross,buijs2011towards,van2010configurable} in cross-organizational process mining attends to compare the same process in different organizations.

 \subsection{Secure Multi-Party Computation}
 \label{sec:mpc}

 Secure Multi-party Computation (MPC)~\cite{GMW} is a cryptographic functionality that allows $n$ parties to cooperatively evaluate $(y_1,\ldots,y_n)=f(x_1,\ldots,x_n)$ for some function $f$, with the $i$-th party contributing the input $x_i$ and learning the output $y_i$, and no party or an allowed coalition of parties learning nothing besides their own inputs and outputs.
 
There exist a few different approaches for constructing MPC protocols. Using the inherently 2-party \emph{garbled circuits} (GC) approach~\cite{yao1982protocols,yao1986generate}, one of the parties encrypts each gate of the Boolean circuit representing $f$, and sends it to the other party, together with the keys corresponding to both parties' inputs. The second party decrypts a part of the representation of each gate, and learns the output of $f$ in the end. Garbled circuit based protocols have small round complexity, but tend to require more bandwidth than some of the next approaches.

Homomorphic encryption (HE)~\cite{elgamal1985public,paillier1999public} can be used to perform computations (mostly linear) on data without seeing it, and threshold homomorphic encryption~\cite{DBLP:conf/eurocrypt/CramerDN01} can be used to build a full MPC protocol. These primitives have been used for privately implementing the Alpha Algorithm for process discovery~\cite{tillem2016privacy,tillem2017mining}. Any computations on encrypted data are enabled by \emph{fully homomorphic encryption} (FHE) schemes~\cite{DBLP:conf/stoc/Gentry09}. While FHE-based approaches minimize the communication between parties, they are very computation-intensive.

Homomorphic secret sharing~\cite{blakley1979safeguarding,shamir1979share} is currently the most common basis for MPC protocols~\cite{CCD,GRR}. In such protocols, the arithmetic or Boolean circuit representing $f$ is evaluated gate-by-gate, constructing secret-shared outputs of gates from their secret-shared inputs. Each evaluation requires some communication between parties (except for addition gates), hence the depth of the circuit determines the round complexity of the protocol. On the other hand, there exist protocols with low communication complexity~\cite{DBLP:journals/ijisec/BogdanovNTW12,DBLP:conf/ccs/ArakiFLNO16,DBLP:conf/crypto/DamgardPSZ12}, allowing the secure computation of quite complex functions $f$, as long as the circuit implementing it has a low multiplicative depth.

The complexity of MPC protocols is heavily dependent on the number of parties jointly performing the computations. Hence the typical deployment of MPC has a relatively small number of \emph{computation parties} (typically just 2 or 3) actually running the protocols for evaluating gates, while an unbounded number of parties may contribute the inputs and/or receive the outputs of the computation~\cite{Kamm-thesis}. There exist frameworks that support such deployments of MPC, they have APIs to simplify the development of privacy-preserving applications~\cite{archer2018keys}. One of such frameworks is Sharemind~\cite{bogdanov2008sharemind}, whose main protocol set~\cite{DBLP:journals/ijisec/BogdanovNTW12} is based on secret-sharing among three computing parties. In this paper, we build on top of Sharemind and its large number of primitive protocols and subroutines~\cite{DBLP:conf/nordsec/BogdanovLT14}, but our techniques are also applicable to other secret sharing based MPC systems. Sharemind framework simplifies our work by offering the SecreC language~\cite{bogdanov2014domain} for programming privacy-preserving applications, abstracting away these details of cryptographic protocols that make sense to be abstracted away.

